\documentclass[11pt]{beamer}
\usetheme[progressbar=frametitle]{metropolis}
%\usetheme{Singapore}
\usepackage[utf8]{inputenc}
\usepackage[spanish]{babel}
\usepackage{fancyvrb}

\author{Lic. Agustina Pesce \\ Lic. Santiago Soler}
\title[Taller de {\LaTeX}]{Taller Introductorio a {\LaTeX}: \\ Cómo producir documentos de Calidad}
\subtitle{Primer Encuentro: Introducción}
%\setbeamercovered{transparent} 
%\setbeamertemplate{navigation symbols}{} 
%\logo{} 
%\institute{} 
\date{}
%\subject{} 
\begin{document}

\maketitle

\section{Bienvenidos al Taller}

\begin{frame}{Taller Introductorio a {\LaTeX}}
\begin{itemize}[<+- | alert@+>] % Con esto hacemos que aparezcan los items de a uno
  \item ¿Cuáles son los objetivos del Taller?
  \item ¿Cómo los vamos a evaluar?
  \item ¿Y cuántas veces vamos a tener que venir?
\end{itemize}

\pause
\textbf{Cronograma Tentativo}


\begin{enumerate}
  \item Primer Encuentro: Introducción
  \item Segundo Encuentro: Artículo
  \item Tercer Encuentro: Bibliografía
  \item Cuarto Encuentro: Libro
  \item Quinto Encuentro: Clase de Documento de Elsevier
  \item Sexto Encuentro: Herramientas varias y cierre del curso
\end{enumerate}

\end{frame}

\begin{frame}{Taller Introductorio a {\LaTeX}}
\begin{center}
¿Y de dónde puedo descargar el material del taller?
\end{center}
%\vspace{1em}
\begin{center}
\textbf{https://santis19.github.io/taller-latex}
\end{center}
\end{frame}

\section{¿Qué es {\LaTeX}?}


\begin{frame}{Editores de texto que todos conocemos}
\pause
\textbf{Editores WYSIWYG (What You See Is What You Get)}
\begin{columns}
\column{0.33\textwidth}
\begin{itemize}
  \item \small{Microsoft Word}
\end{itemize}  
\column{0.33\textwidth}
\begin{itemize}
  \item \small{LibreOffice Writer}
\end{itemize}
\column{0.33\textwidth}
\begin{itemize}
  \item \small{Otros}
\end{itemize}
\end{columns}

\begin{figure}[b]
  \centering
  \includegraphics[width=0.9\textwidth]{figs/libreoffice-writer-sample.png}
\end{figure}
\end{frame}

\begin{frame}{{\LaTeX}: un sistema de composición de textos distinto}
\textbf{Editores WYSIWYM (What You See Is What You Mean)}
\begin{figure}
\centering
\includegraphics[width=0.9\textwidth]{figs/tex-to-pdf.pdf}
\end{figure}
\end{frame}

\begin{frame}{¿Qué es {\LaTeX}?}

\metroset{block=fill}
\begin{block}{Definición}
{\LaTeX} es un sistema de composición de textos, orientado a la creación de documentos escritos que presenten una alta calidad tipográfica.
\end{block}

\begin{itemize}
\item \TeX{} es un sistema de tipografía escrito por Donald E. Knuth en 1978 diseñado para publicar texto y fórmulas matemáticas con gran calidad tipográfica.
\item {\LaTeX} es un conjunto de macros para {\TeX} escrito por Leslie Lamport en 1984 con el propósito de simplificar el manejo de {\TeX}, pero utilizándolo como motor tipográfico.
\end{itemize}
\end{frame}

\begin{frame}{Ventajas de \LaTeX}
\begin{itemize}[<+- | alert@+>]
  \item Produce documentos de altísima calidad de manera automática siguiendo estándares estéticos.
  \item Nos permite dedicar más tiempo al contenido del documento y menos a su edición.
  \item Permite escribir fórmulas matemáticas de altísima calidad.
  \item Estructuras complejas como referencias cruzadas, notas al pie de página, sangría, títulos, tabla de contenidos, bibliografía, etc. son muy sencillas de introducir.
  \item Nos alienta a producir textos bien estructurados, ya que así son los documentos para los que \LaTeX{} está diseñado.
  \item Nos permite dividir nuestro documento en varios archivos. Muy útil a la hora de escribir textos largos como libros o tesis.
  \item Es Software Libre.
\end{itemize}
\end{frame}

\begin{frame}{Software Libre}
\begin{figure}
\centering
\includegraphics[width=0.52\textwidth]{figs/free-software.png}
\end{figure}
\end{frame}


\begin{frame}{Desventajas de \LaTeX}
\metroset{block=fill}
\begin{block}{La curva de aprendizaje posee una pendiente elevada al principio}
Una vez que la superamos nos permite ser más productivos en comparación con un WYSIWYG.
\end{block}

\pause
\begin{block}{No visualizamos el documento definitivo hasta compilar}
Suele incomodar al principio, pero luego nos ahorra tiempo ya que no nos distraemos con la edición del formato.
\end{block}
\end{frame}

\begin{frame}{Desventajas de \LaTeX}
\metroset{block=fill}
\begin{block}{Solemos tener que lidiar con errores de compilación}
Compilando periódicamente nos facilitará la tarea de identificar esos errores en vez de acumularlos.
\end{block}

\pause
\begin{block}{Producir documentos con un formato no estructurado y arbitrario puede ser engorroso}
{\LaTeX} parte de comandos predefinidos, por ende, generar un documento con un formato nuevo requiere reescribir los estilos o bien ``forzar'' nuestro archivo .tex a que genere un documento similar al que deseamos.
\end{block}
\end{frame}

\begin{frame}{\LaTeX{} vs WYSIWYG}
\begin{figure}
\centering
\includegraphics[width=\textwidth]{figs/latex-vs-wysiwyg.pdf}
\end{figure}
\end{frame}


\section{Software Necesario para el Taller}
\begin{frame}[fragile]{Software Necesario para el Taller}
Vamos a necesitar una distribución de \LaTeX{}, un editor y un administrador de bases de datos bibliográficas.

\vspace{1em}

\begin{columns}
\column{0.5\textwidth}
\metroset{block=fill}
\begin{block}{\small GNU/Linux}
\begin{footnotesize}
\begin{description}[fontsize=\small]
  \item[Texlive:] Distribución de \LaTeX{}
  \item[TexMaker:] Editor \LaTeX{}
  \item[JabRef:] Administrador de Base de Datos Bibliográficas
\end{description}
\end{footnotesize}
\end{block}

\column{0.5\textwidth}
\metroset{block=fill}
\begin{block}{\small Windows}
\begin{footnotesize}
\begin{description}[fontsize=\small]
  \item[MikTex:] Distribución de \LaTeX{}
  \item[TexMaker:] Editor \LaTeX{}
  \item[JabRef:] Administrador de Base de Datos Bibliográficas
\end{description}
\end{footnotesize}
\end{block}
\end{columns}

\end{frame}

\section{Hola Mundo en \LaTeX}

\end{document}